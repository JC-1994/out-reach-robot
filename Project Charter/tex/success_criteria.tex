%The success criteria are enumerated effects outside of the development of the solution (i.e., NOT specific project requirements) that can be observed and measured to quantify "what success looks like". The key is to focus on specific expected benefits beginning immediately after the project is delivered and projecting forward into the future. Bullet lists should be used to itemize each success criterion, and each item should have a time frame and some sort of quantifiable measurement.

%One way to list the success criteria is to use lists for different time frames. Here is a short example:
%\\
%\\
Upon completion of the  system, we expect the following success indicators to be observed:
\begin{itemize}
  \item System is able to function for at least 60 minutes of activity (copying what someone draws) without the need for reboot.
  \item System is able to copy what the user draws with 70 percent accuracy.
\end{itemize}

Within 1 months before the delivery date, we expect the following success indicators to be observed:
\begin{itemize}
  \item Application is able to communicate with robotic arm. 
  \item System is able to function for at least 5 minutes of activity (copying what someone draws) without the need for reboot.
  \item System is able to copy what user draws with 30 percent accuracy
\end{itemize}


Within 2 months before the delivery date, we expect the following success indicators to be observed:
\begin{itemize}
  \item Robotic arm is complete and able to draw. 
  \item Application is complete.
\end{itemize}

Within 3 months before the delivery date, we expect the following success indicators to be observed:
\begin{itemize}
  %\item A 10\% reduction in operating costs
  \item Application on tablet is able to run, but not 100 percent complete.
  \item Robotic arm is able to draw basic lines on paper with basic movements.
  %\item 30\% reduction in average transaction time
  %\item 20\% increase in mean time to failure (MTTF)
\end{itemize}





~\\
~\\
%NOTE: The vision, mission, and success criteria, when combined, should occupy \underline{EXACTLY ONE FULL PAGE}. This can be individually distributed as an agile project charter or executive summary when necessary. 
%\\
%NOTE: Throughout the document, remove and replace all instruction text with your own material. 
